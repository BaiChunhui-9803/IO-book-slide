%-------------------------------------------------------------------------------------------
\documentclass[aspectratio=169,UTF8,11pt]{ctexbeamer}

%%%%%%%%%%%%%%%%%%%%%%%%%%%%%
\usepackage{colortbl}
\usepackage{color}
\usepackage{booktabs}
\usepackage{threeparttable}
\usepackage{hyperref}
\usepackage{bm}
\usepackage{amsmath}
\usepackage[ruled,vlined]{algorithm2e}
%\usepackage{babel}
%%%%%%%%%%%%%%%%%%%%%%%%%%%%%

\mode<presentation> {
\usetheme{Madrid}
%\setbeamertemplate{footline} % To remove the footer line in all slides uncomment this line
\setbeamertemplate{footline}[frame number] % To replace the footer line in all slides with a simple slide count uncomment this line
\setbeamercolor{page number in head/foot}{fg=blue}
\setbeamertemplate{navigation symbols}{} % To remove the navigation symbols from the bottom of all slides uncomment this line
}

% User Defined Block %%%%%%%%%%%%%%%%%%%%%%%%%%%%%%%%%%%%%%%%%%%%%%%%%%%%%%%%
\usepackage{setspace}
\definecolor{orange}{rgb}{1,0.5,0}
\definecolor{aa}{RGB}{34,139,34}
\definecolor{lightblue}{rgb}{0,0.85,0.9}
\definecolor{darkblue}{rgb}{0,0.7,1}

\definecolor{hanblue}{rgb}{0.27, 0.42, 0.81}
\definecolor{indiagreen}{rgb}{0.07, 0.53, 0.03}
\definecolor{indianred}{rgb}{0.8, 0.36, 0.36}
\definecolor{indianyellow}{rgb}{0.89, 0.66, 0.34}
\definecolor{babypink}{rgb}{0.96, 0.76, 0.76}
\definecolor{ao(english)}{rgb}{0.0, 0.5, 0.0}
\setbeamerfont{block title}{size=\normalsize}
\setbeamerfont{block body}{size=\small}

\newenvironment<>{blueblock}[1]{%
  \setbeamercolor{block title}{fg=white,bg=hanblue}%
  \begin{block}#2{#1}}{\end{block}}

\newenvironment<>{greenblock}[1]{%
  \setstretch{1.3}\setbeamercolor{block title}{fg=white,bg=indiagreen}%
  \begin{block}#2{#1}}{\end{block}}

\newenvironment<>{redblock}[1]{%
  \setstretch{1.3}\setbeamercolor{block title}{fg=white,bg=indianred}%
  \begin{block}#2{#1}}{\end{block}}

\newenvironment<>{yellowblock}[1]{%
  \setstretch{1.3}\setbeamercolor{block title}{fg=white,bg=indianyellow}%
  \begin{block}#2{#1}}{\end{block}}

%----------------------------------------------------------------------------------------
%	PACKAGES
%----------------------------------------------------------------------------------------
\usepackage{graphicx} % Allows including images
%\usepackage{tikz}
%\usetikzlibrary{shapes.geometric, arrows}
\usepackage{listings}
\lstset{language=C++,
    columns=flexible,
   % basicstyle=\scriptsize\ttfamily,                                      % 设定代码字体、大小4
    basicstyle=\footnotesize\ttfamily,
    %numbers=left,xleftmargin=2em,framexleftmargin=2em,                   % 在左侧显示行号
    %numberstyle=\color{darkgray},                                        % 设定行号格式
    keywordstyle=\color{blue},                                            % 设定关键字格式
    commentstyle=\color{ao(english)},                                     % 设置代码注释的格式
    stringstyle=\color{brown},                                            % 设置字符串格式
    %showstringspaces=false,                                              % 控制是否显示空格
	%frame=lines,                                                         % 控制外框
    breaklines,                                                           % 控制是否折行
    postbreak=\space,                                                     % 控制折行后显示的标识字符
    breakindent=5pt,                                                      % 控制折行后缩进数量
    emph={size\_t,array,deque,list,map,queue,set,stack,vector,string,pair,tuple}, % 非内置类型
    emphstyle={\color{teal}},
    escapeinside={(*@}{@*)},
}
%---------------------------------------------------------------------------------------------------

%%%%%%%%%%%%%%%%%%%%%%%%%%%%%%%%%%%%%%%%%%%%%%%%%%%%%%%%%%%%%%%%%%%%%%%%%%%%%%%%%%%%%%%%%%%%%%
\title[\textit{智能优化与最优化方法}]{15.4 Games}
\author[李长河]{李长河} % Your name
\institute[CUG] % Your institution as it will appear on the bottom of every slide, may be shorthand to save space
{
中国地质大学(武汉)自动化学院\\ % Your institution for the title page
\medskip
\textit{lichanghe@cug.edu.cn} % Your email address
}
\date{} % Date, can be changed to a custom date
%%%%%%%%%%%%%%%%%%%%%%%%%%%%%%%%%%%%%%%%%%%%%%%%%%%%%%%%%%%%%%%%%%%%%%%%%%%%%%%%%%%%%%%%%%%%%%

\usefonttheme[onlymath]{serif}
\begin{document}
\maketitle
\begin{frame}[noframenumbering]           %beamer里重要的概念,每个frame定义一张page
\centering
{\large 李长河 \vspace{0.5cm} \\自动化学院710 \vspace{0.5cm}\\ lichanghe@cug.edu.cn}
\end{frame}

%-----------------------------------------------------------


\addtocounter{framenumber}{-1}
%---------------------------------------------------------------------------------------------

\begin{frame}{Contents}
	\tableofcontents
\end{frame}

%%%%%%%%%%%%%%%%%%%%%%%%%%%%%%%%%%%%%%%%%%%%%%%%%%%%%%%%%%%%%%%%%%%%%%%%%%%%%%%%%%%%%%%%%%%%%%
\section{15.4 Games}
\begin{frame}
  {Games}
  \begin{block}{}
    For AI research, researchers attempt to devise computational systems that would exhibit aspects of \alert{human intelligence} and achieve \alert{human-level} problem solving or decision making abilities.
  \end{block}
  \begin{block}{}
    The highly formalized, symbolic representation allowed AI to success in many cases.
  \end{block}
  \begin{greenblock}{}
    Naturally, games, especially board games, have been a popular domain for AI researches as they are \alert{formal} and \alert{highly constrained}, yet \alert{complex}, decision making environments. 
  \end{greenblock}
\end{frame}

%%%%%%%%%%%%%%%%%%%%%%%%%%%%%%%%%%%%%%%%%%%%%%%%%%%%%%%%%%%%%%%%%%%%%%%%%%%%%%%%%%%%%%%%%%%%%%
\subsection{15.4.1 Applications}
\begin{frame}
  \tableofcontents[currentsection,currentsubsection]
\end{frame}

\begin{frame}
  {Games\small{-Applications}}
  \begin{yellowblock}{}
    There are two main applications of intelligent optimization on games:
    \begin{itemize}
      \item Game playing
      \item Content generation
    \end{itemize}
  \end{yellowblock}
\end{frame}

\begin{frame}
  {Games\small{-Applications}}
  \begin{block}{Game Playing}
    For game playing, there are two main ways to use the intelligent optimization:
    \begin{itemize}
      \item Offline optimization: Optimizing the parameters of a pre-defined controller, then using the optimized controller to play the game.
      \item Online optimization (or evolutionary planning): Optimizing the actions to be deployed while the game is running.
    \end{itemize}
  \end{block}
\end{frame}

\begin{frame}
  {Games\small{-Applications}}
  \begin{block}{In the offline optimization}
    The intelligent optimization algorithm can be combined with many other methods. In particular, using EAs to evolve: 
    \begin{itemize}
      \item the weights and/or topology of \alert{neural networks}, 
      \item or \alert{programs}, typically structured as expression trees (e.g., genetic programming), 
      \item or some other \alert{models}, such as potential field.
    \end{itemize}
    This fitness evaluation consists of using the neural network, program or other models to play the game, and using the \alert{result (e.g., score) as a fitness function}.
  \end{block}
\end{frame}

\begin{frame}
  {Games\small{-Applications}}
  \begin{block}{In the online optimization}
    The basic idea is to optimize the \alert{action sequence} you want to deploy from now on. Evaluating such an action sequence is done by taking all the actions of the sequence in \alert{simulation}, and observing the results after taking all those actions.
  \end{block}
  \begin{block}{}
    The online optimization has been applied on several types of games such as: 
    \begin{itemize}
      \item arcade game
      \item turn-based strategy game
      \item real-time strategy game
    \end{itemize}
  \end{block}
\end{frame}

\begin{frame}
  {Games\small{-Applications}}
  \begin{block}{Content Generation}
    Procedural content generation refers to the methods which generate game content either automatically or with only limited human input, such as: 
    \begin{itemize}
      \item levels
      \item maps
      \item game rules
    \end{itemize}
  \end{block}
  \begin{blueblock}{}
    The search-based methods are widely used in this area because we will eventually get the solution we want if we keep iterating and tweaking solutions by keeping the good changes and discarding the bad changes. 
  \end{blueblock}
\end{frame}
%%%%%%%%%%%%%%%%%%%%%%%%%%%%%%%%%%%%%%%%%%%%%%%%%%%%%%%%%%%%%%%%%%%%%%%%%%%%%%%%%%%%%%%%%%%%%%
\subsection{15.4.2 A Simple Example of Real-time Strategy Game}
\begin{frame}
  \tableofcontents[currentsection,currentsubsection]
\end{frame}

\begin{frame}
  {Games\small{-A Simple Example of Real-time Strategy Game}}
  \begin{block}{}
    There is a consistent pursuing for our human to obtain the artificial intelligence.
  \end{block}
\end{frame}

\end{document}