%-------------------------------------------------------------------------------------------
\documentclass[aspectratio=169,UTF8,11pt]{ctexbeamer}

%%%%%%%%%%%%%%%%%%%%%%%%%%%%%
\usepackage{colortbl}
\usepackage{color}
\usepackage{booktabs}
\usepackage{threeparttable}
\usepackage{hyperref}
\usepackage{bm}
\usepackage{amsmath}
\usepackage[ruled,vlined]{algorithm2e}
%\usepackage{babel}
%%%%%%%%%%%%%%%%%%%%%%%%%%%%%

\mode<presentation> {
\usetheme{Madrid}
%\setbeamertemplate{footline} % To remove the footer line in all slides uncomment this line
\setbeamertemplate{footline}[frame number] % To replace the footer line in all slides with a simple slide count uncomment this line
\setbeamercolor{page number in head/foot}{fg=blue}
\setbeamertemplate{navigation symbols}{} % To remove the navigation symbols from the bottom of all slides uncomment this line
}

% User Defined Block %%%%%%%%%%%%%%%%%%%%%%%%%%%%%%%%%%%%%%%%%%%%%%%%%%%%%%%%
\usepackage{setspace}
\definecolor{orange}{rgb}{1,0.5,0}
\definecolor{aa}{RGB}{34,139,34}
\definecolor{lightblue}{rgb}{0,0.85,0.9}
\definecolor{darkblue}{rgb}{0,0.7,1}

\definecolor{hanblue}{rgb}{0.27, 0.42, 0.81}
\definecolor{indiagreen}{rgb}{0.07, 0.53, 0.03}
\definecolor{indianred}{rgb}{0.8, 0.36, 0.36}
\definecolor{indianyellow}{rgb}{0.89, 0.66, 0.34}
\definecolor{babypink}{rgb}{0.96, 0.76, 0.76}
\definecolor{ao(english)}{rgb}{0.0, 0.5, 0.0}
\setbeamerfont{block title}{size=\normalsize}
\setbeamerfont{block body}{size=\small}

\newenvironment<>{blueblock}[1]{%
  \setbeamercolor{block title}{fg=white,bg=hanblue}%
  \begin{block}#2{#1}}{\end{block}}

\newenvironment<>{greenblock}[1]{%
  \setstretch{1.3}\setbeamercolor{block title}{fg=white,bg=indiagreen}%
  \begin{block}#2{#1}}{\end{block}}

\newenvironment<>{redblock}[1]{%
  \setstretch{1.3}\setbeamercolor{block title}{fg=white,bg=indianred}%
  \begin{block}#2{#1}}{\end{block}}

\newenvironment<>{yellowblock}[1]{%
  \setstretch{1.3}\setbeamercolor{block title}{fg=white,bg=indianyellow}%
  \begin{block}#2{#1}}{\end{block}}

%----------------------------------------------------------------------------------------
%	PACKAGES
%----------------------------------------------------------------------------------------
\usepackage{graphicx} % Allows including images
%\usepackage{tikz}
%\usetikzlibrary{shapes.geometric, arrows}
\usepackage{listings}
\lstset{language=C++,
    columns=flexible,
   % basicstyle=\scriptsize\ttfamily,                                      % 设定代码字体、大小4
    basicstyle=\footnotesize\ttfamily,
    %numbers=left,xleftmargin=2em,framexleftmargin=2em,                   % 在左侧显示行号
    %numberstyle=\color{darkgray},                                        % 设定行号格式
    keywordstyle=\color{blue},                                            % 设定关键字格式
    commentstyle=\color{ao(english)},                                     % 设置代码注释的格式
    stringstyle=\color{brown},                                            % 设置字符串格式
    %showstringspaces=false,                                              % 控制是否显示空格
	%frame=lines,                                                         % 控制外框
    breaklines,                                                           % 控制是否折行
    postbreak=\space,                                                     % 控制折行后显示的标识字符
    breakindent=5pt,                                                      % 控制折行后缩进数量
    emph={size\_t,array,deque,list,map,queue,set,stack,vector,string,pair,tuple}, % 非内置类型
    emphstyle={\color{teal}},
    escapeinside={(*@}{@*)},
}
%---------------------------------------------------------------------------------------------------

%%%%%%%%%%%%%%%%%%%%%%%%%%%%%%%%%%%%%%%%%%%%%%%%%%%%%%%%%%%%%%%%%%%%%%%%%%%%%%%%%%%%%%%%%%%%%%
\title[\textit{智能优化与最优化方法}]{13 Large-scale Global Optimization}
\author[李长河]{李长河} % Your name
\institute[CUG] % Your institution as it will appear on the bottom of every slide, may be shorthand to save space
{
中国地质大学(武汉)自动化学院\\ % Your institution for the title page
\medskip
\textit{lichanghe@cug.edu.cn} % Your email address
}
\date{} % Date, can be changed to a custom date
%%%%%%%%%%%%%%%%%%%%%%%%%%%%%%%%%%%%%%%%%%%%%%%%%%%%%%%%%%%%%%%%%%%%%%%%%%%%%%%%%%%%%%%%%%%%%%

\usefonttheme[onlymath]{serif}
\begin{document}
\maketitle
\begin{frame}[noframenumbering]           %beamer里重要的概念,每个frame定义一张page
\centering
{\large 李长河 \vspace{0.5cm} \\自动化学院710 \vspace{0.5cm}\\ lichanghe@cug.edu.cn}
\end{frame}

%-----------------------------------------------------------


\addtocounter{framenumber}{-1}
%---------------------------------------------------------------------------------------------

\begin{frame}
  {Large-scale Global Optimization}
  \begin{block}{Complex Optimization Problems}
    \begin{itemize}
    \item One of the manifestations of complexity is that \alert{the increase in the number of variables} causes the \alert{dimensionality disaster}.
    \item Large-scale global optimization (LSGO) problems usually contains more than 1000 variables.
    \end{itemize}
  \end{block}
  \pause
  \begin{greenblock}{Application Examples}
    \begin{itemize}
      \item The large-scale power system design
      \item Vehicle routing problems
      \item Genetic identification
      \item Inverse chemical kinetics problem
    \end{itemize}
  \end{greenblock}
\end{frame}

\begin{frame}{Contents}
	\tableofcontents
\end{frame}

%%%%%%%%%%%%%%%%%%%%%%%%%%%%%%%%%%%%%%%%%%%%%%%%%%%%%%%%%%%%%%%%%%%%%%%%%%%%%%%%%%%%%%%%%%%%%%
\section{13.1 Large-scale Global Optimization Problems}
%%%%%%%%%%%%%%%%%%%%%%%%%%%%%%%%%%%%%%%%%%%%%%%%%%%%%%%%%%%%%%%%%%%%%%%%%%%%%%%%%%%%%%%%%%%%%%
\subsection{13.1.1 Definition}
\begin{frame}
  \tableofcontents[currentsubsection]
\end{frame}

\begin{frame}
  {Large-scale Global Optimization Problems\small{-Definition}}
  \begin{block}{Complex Optimization Problems}
    \begin{itemize}
    \item To effectively solve the large-scale global optimization problem, we normally do not search in the original search space.
    \item According to the correlation between variables, the problem can be \alert{separable}, \alert{partially separable} or \alert{non-separable}.
    \end{itemize}
  \end{block}

  \only<1>{
  \begin{greenblock}
    {Fully separable function}
    $f(\bm{x})$ is a {\bf fully separable} function iff (if and only if) \index{F!fully separable}
    \begin{align}
    \min / \max f(\bm{x})=\left(f\left(x_{1}\right), f\left(x_{2}\right),\cdots,  f\left(x_{n}\right)\right) \tag{1}\label{eq13-1}
    \end{align}
    where $\bm{x} = \left(x_{1}, x_{2}, \cdots, x_{n}\right)$ is a D-dimensional decision vector.
  \end{greenblock}
  }

  \only<2>{
  \begin{greenblock}
    {Partially separable function}
    $f(\bm{x})$ is a {\bf partially separable} function with $m$ independent subcomponents iff \index{P!partially separable}
    \begin{align}
    \min / \max f(\bm{x})=\left(f\left(\bm{x_{1}}, \cdots\right),\cdots,  f\left(\cdots,\bm{x_{m}}\right)\right) \tag{2}\label{eq13-2}
    \end{align}
    where $\bm{x_{1}}, \cdots, \bm{x_{m}}$ are disjoint sub-vectors of $\bm{x}$ and $2 \leq m<n$.
  \end{greenblock}
  }

  \only<3>{
    \begin{greenblock}
      {Fully non-separable function}
      $f(\bm{x})$ is a {\bf fully non-separable} function, if every pair of its decision variables interacts with each other.\index{F!fully non-separable}
    \end{greenblock}
  }

  \only<4>{
    \begin{greenblock}
      {Partially additively separable function}
      $f(\bm{x})$ is a {\bf partially additively separable} function iff\index{P!partially additively separable}
      \begin{align}
      f(\bm{x})=\sum_{i=1}^{m} f_{i}\left(\bm{x_{i}}\right) \tag{3}\label{eq13-1-4}
      \end{align}
      where $\bm{x_{i}}$ are mutually exclusive decision vectors of $f_{i}$ ; $m$ is the number of separable subcomponents.
    \end{greenblock}
  }
  
\end{frame}


\subsection{13.1.2 Difficulties}
\begin{frame}
  \tableofcontents[currentsubsection]
\end{frame}

\begin{frame}
  {Large-scale Global Optimization Problems\small{-Difficulties}}
  \begin{block}
    {Difficulties}
    {\bf Non-separable} and {\bf overlapping} functions are the most difficult to solve:
    \begin{itemize}
      \item The search space generally \alert{increases exponentially}.
      \item The algorithm is easy to \alert{fall into local optima}.
      \item The objective function generally has the characteristics of \alert{nonlinear}, \alert{non-convex}, \alert{multi-modal} and \alert{non-differentiable}.
      \item Variables are \alert{partially separable} or \alert{completely non-separable}.
    \end{itemize}
    {\bf It is also challenging for EAs to explore the entire search space effectively.}
  \end{block}
  \pause
  \begin{greenblock}{Two branches of metaheuristics}
    \begin{itemize}
      \item Space decomposition methods (co-evolution).
      \item Non-decomposition-based methods.
    \end{itemize}
  \end{greenblock}
\end{frame}
%%%%%%%%%%%%%%%%%%%%%%%%%%%%%%%%%%%%%%%%%%%%%%%%%%%%%%%%%%%%%%%%%%%%%%%%%%%%%%%%%%%%%%%%%%%%%%
\section{13.2 Co-evolution Methods}
\begin{frame}
  {Co-evolution Methods}
  \begin{block}{Methods Mechanism}
    \begin{itemize}
      \item Decompose the large-scale global optimization problem into several \alert{low-dimensional sub-problems}.
      \item Use the \alert{divide-and-conquer method} to solve the large-scale optimization problem.
    \end{itemize}
  \end{block}
  \pause
  \begin{redblock}{Challenge}
    \begin{itemize}
      \item The interaction between variables in indivisible problems.
      \item It will have a great influence on the optimization efficiency of the algorithm.
    \end{itemize}
  \end{redblock}
  \pause
  \begin{yellowblock}{Solution}
    Most decomposition methods try to \alert{identify} the interacting variables and \alert{assign} them to the same in the sub-question.
  \end{yellowblock}
\end{frame}
%%%%%%%%%%%%%%%%%%%%%%%%%%%%%%%%%%%%%%%%%%%%%%%%%%%%%%%%%%%%%%%%%%%%%%%%%%%%%%%%%%%%%%%%%%%%%%
\subsection{13.2.1 Co-operative Co-evolution}
\begin{frame}
  \tableofcontents[currentsubsection]
\end{frame}

\begin{frame}
  {Co-evolution Methods\small{-Co-operative Co-evolution}}  
  \begin{block}{Co-operative co-evolutionary (CC) approach}
    Potter and De Jong proposed the co-operative co-evolutionary (CC) approach in 1994 to improve the performance of GAs.
  \end{block}
  \pause
  \begin{greenblock}{The framework of classic CC algorithm}
    \begin{itemize}
      \item Problem decomposition.
      \item Subcomponent optimization.
      \item Cooperative combination.
    \end{itemize}
  \end{greenblock}
  \pause
  \begin{yellowblock}{Categories}
    Based on different grouping strategies, the co-evolution algorithms used to solve LSGO problems are mainly divided into two categories: \alert{static grouping methods} and \alert{dynamic grouping methods}.
  \end{yellowblock}
\end{frame}

\subsection{13.2.2 Static Grouping}
\begin{frame}
  \tableofcontents[currentsubsection]
\end{frame}

\begin{frame}
  {Co-evolution Methods\small{-Static Grouping}}
  \begin{block}{Limitations of CC algorithm}
    \begin{itemize}
      \item Being tested only for the maximum dimension of 30.
      \item Being less effective than classical GAs in non-separable problems.
    \end{itemize}
  \end{block}
  \pause
  \begin{greenblock}{Solutions}
    \begin{itemize}
      \item Liu et al combined CC framework with FEP (fast evolutionary programming) to solve the problem of 100–1000 dimensional continuous optimization.
      \item Scholars tried to combine the CC idea with swarm intelligence-based algorithms,
      such as CPSO-SK and CPSO-HK.
    \end{itemize}
  \end{greenblock}
  \pause
  \begin{yellowblock}{Limitations of Static grouping}
    Only effective in low-dimensional problems.
  \end{yellowblock}
\end{frame}


\subsection{13.2.3 Dynamic Grouping}
\begin{frame}
  \tableofcontents[currentsubsection]
\end{frame}

\begin{frame}
  {Co-evolution Methods\small{-Dynamic Grouping}}
  \begin{block}{Relationships between static and dynamic Grouping}
    \begin{itemize}
      \item They both try to detect the interacting relationship between variables and assign the variables that interacts each other to the same sub-component.
      \item In static grouping, the number of sub-components $(k)$ is fixed, while in dynamic grouping, the structure of sub-components can be dynamically adjusted.
    \end{itemize}
  \end{block}
  \pause
  \begin{greenblock}{Categories}
    Dynamic grouping methods can be divided into two categories based on the substitution of variables in group components: \alert{random dynamic grouping} and \alert{learning-based dynamic grouping}.
  \end{greenblock}
\end{frame}

\begin{frame}
  \begin{greenblock}{DECC-G algorithm}
    \begin{algorithm}[H]
      \caption{DECC-G algorithm}
      \label{alg13-2-2}
      \While{$termination \, criterion \,is\, not\, fulfilled$}{
      Set $i=0$\;
      The $n$-dimensional object vector is randomly divided into $m$ $s$-dimensional subcomponents\;
      \While{$i<m$}{
      $i++$\;
      Evolve the $i$th subcomponent with a certain EA\;
      }
      Assign a weight vector for each subcomponent\;
      Optimize them via a certain EA\;
      }
      \end{algorithm}
  \end{greenblock}
\end{frame}



%%%%%%%%%%%%%%%%%%%%%%%%%%%%%%%%%%%%%%%%%%%%%%%%%%%%%%%%%%%%%%%%%%%%%%%%%%%%%%%%%%%%%%%%%%%%%%
\section{13.3 Non-decomposition-based Methods}
%%%%%%%%%%%%%%%%%%%%%%%%%%%%%%%%%%%%%%%%%%%%%%%%%%%%%%%%%%%%%%%%%%%%%%%%%%%%%%%%%%%%%%%%%%%%%%
\subsection{13.3.1 PSO-based Algorithms}
\begin{frame}
  \tableofcontents[currentsubsection]
\end{frame}

\subsection{13.3.2 EDA-based Algorithms}
\begin{frame}
  \tableofcontents[currentsubsection]
\end{frame}

\subsection{13.3.3 DE-based Algorithms}
\begin{frame}
  \tableofcontents[currentsubsection]
\end{frame}





\end{document}
